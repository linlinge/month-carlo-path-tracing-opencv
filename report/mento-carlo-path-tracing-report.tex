\documentclass[10pt]{article}
\usepackage[UTF8]{ctex}
\usepackage{picinpar,graphicx,bm}
\usepackage{booktabs}
\usepackage{diagbox}
\usepackage{float}
\usepackage{multirow}

\usepackage{listings}
\usepackage{xcolor}
% 定义可能使用到的颜色
\definecolor{CPPLight}  {HTML} {686868}
\definecolor{CPPSteel}  {HTML} {888888}
\definecolor{CPPDark}   {HTML} {262626}
\definecolor{CPPBlue}   {HTML} {4172A3}
\definecolor{CPPGreen}  {HTML} {487818}
\definecolor{CPPBrown}  {HTML} {A07040}
\definecolor{CPPRed}    {HTML} {AD4D3A}
\definecolor{CPPViolet} {HTML} {7040A0}
\definecolor{CPPGray}  {HTML} {B8B8B8}
\lstset{
    columns=fixed,    
   % numbers=left,                                        % 在左侧显示行号
    frame=none,                                          % 不显示背景边框
    backgroundcolor=\color[RGB]{245,245,244},            % 设定背景颜色
    keywordstyle=\color[RGB]{40,40,255},                 % 设定关键字颜色
    numberstyle=\footnotesize\color{darkgray},           % 设定行号格式
    commentstyle=\it\color[RGB]{0,96,96},                % 设置代码注释的格式
    stringstyle=\rmfamily\slshape\color[RGB]{128,0,0},   % 设置字符串格式
    showstringspaces=false,                              % 不显示字符串中的空格
    language=c++,                                        % 设置语言
    morekeywords={alignas,continute,friend,register,true,alignof,decltype,goto,
    reinterpret_cast,try,asm,defult,if,return,typedef,auto,delete,inline,short,
    typeid,bool,do,int,signed,typename,break,double,long,sizeof,union,case,
    dynamic_cast,mutable,static,unsigned,catch,else,namespace,static_assert,using,
    char,enum,new,static_cast,virtual,char16_t,char32_t,explict,noexcept,struct,
    void,export,nullptr,switch,volatile,class,extern,operator,template,wchar_t,
    const,false,private,this,while,constexpr,float,protected,thread_local,
    const_cast,for,public,throw,std,size_t,__global__,__device__,__host__},
    emph={map,set,multimap,multiset,unordered_map,unordered_set,
    unordered_multiset,unordered_multimap,vector,string,list,deque,
    array,stack,forwared_list,iostream,memory,shared_ptr,unique_ptr,
    random,bitset,ostream,istream,cout,cin,endl,move,default_random_engine,
    uniform_int_distribution,iterator,algorithm,functional,bing,numeric,},
    emphstyle=\color{CPPViolet}, 
    frame=shadowbox,
    basicstyle=\footnotesize\ttfamily,
    tabsize=4,
}

\newcommand{\tabincell}[2]{\begin{tabular}{@{}#1@{}}#2\end{tabular}}  


%layout
\usepackage{calc} 
\setlength\textwidth{7in} 
\setlength\textheight{9in} 
\setlength\oddsidemargin{(\paperwidth-\textwidth)/2 - 1in}
\setlength\topmargin{(\paperheight-\textheight -\headheight-\headsep-\footskip)/2 - 1.5in}


\title{计算机图形学 \hspace{2pt}\hspace{2pt} \begin{large}----- \hspace{2pt} 蒙特卡罗光线跟踪算法 \end{large} }
\author{11821095 葛林林}
\begin{document}
\maketitle
\section{预备知识}
\subsection{光照类型}
\begin{itemize}
\item[(1)]{\bm{环境光}}:环境光无处不在,无论表面的法向如何,明暗程度都是一致的。
\item[(2)]{\bm{点光源}}:光源来自某个点,且向四面八方辐射。
\item[(3)]{\bm{平行光}}:又称为镜面光,这种光是互相平行的。从手电筒或者太阳出来的光都可以被看做平行光。
\item[(4)]{\bm{聚光灯}}:这种光源的光线从一个锥体中射出,在被照射的物体上产生聚光的效果。使用这种光源需要指定光的射出方向以及锥体的顶角$\alpha$。
\end{itemize}
\subsection{obj文件介绍}
obj文件并不考虑物体的大小,所以不同的物体读入的坐标范围可能变化很大,因此为了显示的方便需将其转为当前绘制坐标系中。
\begin{itemize}
\item[(1)]{库相关}
\begin{itemize}
\item[•]{$\bm{mtllib}$ xxx} \mbox{} \\
材料库。
\item[•]{$\bm{usemtl}$ xxx} \mbox{} \\
代表使用xxx类型的材质。
\end{itemize}

\item[(2)]{组相关} 
\begin{itemize}
\item[•]{$\bm{g}$ xxx} \mbox{} \\
表示组,将xxx标签之后的多边形组成一个整体。
\item[•]{$\bm{s}$ xxx} \mbox{} \\
光滑组:加入光滑组之后能够让在同一组的多边形之间连接更为光滑,其中“s off”代表关闭光滑组。
\end{itemize}

\item[(3)]{坐标相关}
\begin{itemize}
\item[•]{$\bm{vt} \hspace{5pt}tu \hspace{5pt} tv$} \mbox{} \\
代表纹理坐标。
\item[•]{$\bm{vn} \hspace{5pt} nx \hspace{5pt} ny \hspace{5pt} nz$} \mbox{} \\
法向量的表示。
\item[•]{$\bm{f} \hspace{5pt} v/vt/vn \hspace{5pt} v/vt/vn \hspace{5pt} v/vt/vn$}\mbox{} \\
表示多边形,格式为“f 顶点索引 / 纹理坐标索引 / 顶点法向量索引”。
\item[•]{$\bm{v} \hspace{5pt} x\hspace{5pt}y\hspace{5pt}z$} \mbox{} \\
顶点以$v$开头后面跟着该顶点的$x,y,z$三轴坐标。
\end{itemize}

\end{itemize}

\subsection{mtl文件介绍}
mtl文件是用来描述文件材质的一个文件,描述的是物体的材质信息,ASCII存储,任何文本编辑器可以将其打开和编辑。一个.mtl文件可以包含一个或多个材质定义,对于每个材质都有其颜色,纹理和反射贴图的描述,应用于物体的表面和顶点。描述的是物体的材质信息,ASCII存储,任何文本编辑器可以将其打开和编辑。一个.mtl文件可以包含一个或多个材质定义,对于每个材质都有其颜色,纹理和反射贴图的描述,应用于物体的表面和顶点。:
\begin{itemize}
\item[(4)]{\bm{格式}:\bm{$Ka} \hspace{10pt} r \hspace{10pt} g \hspace{10pt} b$} \\
\bm{示例}:$\bm{Ks} \hspace{10pt} 0.588 \hspace{10pt} 0.588 \hspace{10pt} 0.588$\\
\bm{描述}:环境反射,用RGB颜色值来表示,g和b两参数是可选的,如果只指定了r的值,则g和b的值都等于r的值。三个参数一般取值范围为[0.0,1.0],在此范围外的值则相应的增加或减少反射率;

\item[(5)]{\bm{格式}:\bm{$Kd} \hspace{10pt} r \hspace{10pt} g \hspace{10pt} b$}\\
\bm{示例}:$\bm{Kd} \hspace{10pt} 0.65 \hspace{10pt} 0.65 \hspace{10pt} 0.65$\\
\bm{描述}:漫反射,$r g b$代表了RGB值,范围为[0,1]。其中$g$和$b$是可选的,如果未设置这两个值,则$g,b$的值与$r$的值相同。

\item[(6)]{\bm{格式}:\bm{$Ks} \hspace{10pt} r \hspace{10pt} g \hspace{10pt} b$}\\
\bm{示例}:$\bm{Ks} \hspace{10pt} 0.65 \hspace{10pt} 0.65 \hspace{10pt} 0.65$\\
\bm{描述}:镜面反射,$r g b$代表了RGB值,范围为[0,1]。其中$g$和$b$是可选的,如果未设置这两个值,则$g,b$的值与$r$的值相同。


\item[(6)]{\bm{格式}:\bm{$Tf} \hspace{10pt} r \hspace{10pt} g \hspace{10pt} b$}\\
\bm{示例}:$\bm{Tf} \hspace{10pt} 0.65 \hspace{10pt} 0.65 \hspace{10pt} 0.65$\\
\bm{描述}:代表了透射滤波,任何光线穿透该物体时可以利用该参数进行透射滤波,该参数只让指定颜色的光线穿透物体。例如Tf 0 1 0,只允许所有的绿色光线穿透,而所有的红色和绿色光线则不能够穿透。$r g b$代表了RGB值,范围为[0,1]。其中$g$和$b$是可选的,如果未设置这两个值,则$g,b$的值与$r$的值相同。

\item[(1)]{\bm{格式}:\bm{$illum} \hspace{10pt} number$} \\
\bm{示例}:\bm{$illum} \hspace{10pt} 2$\\
\bm{描述}:指定了光照模型,这些模型总共分成11种,具体的定义如下:
\begin{itemize}
\item[0]{这是一个常数照明模型,将Kd作为材料的颜色,即}
$$color=Kd$$

\item[1]{这是一个漫反射照明模型,}
$$color=KaIa+ Kd { SUM j=1..ls, (N * Lj)Ij }$$

\item[2]{这是}
$$color = KaIa 
 	+ Kd { SUM j=1..ls, (N*Lj)Ij }
 	+ Ks { SUM j=1..ls, ((H*Hj)^Ns)Ij }$$

\item[3]{这是}
$$color = KaIa
 	+ Kd { SUM j=1..ls, (N*Lj)Ij }
 	+ Ks ({ SUM j=1..ls, ((H*Hj)^Ns)Ij } + Ir)$$

\item[4]{漫反射和镜面反射光照模型,该模型用来仿真出玻璃的效果。}
$$color = KaIa + Kd { SUM j=1..ls, (N*Lj)Ij } + Ks ({ SUM j=1..ls, ((H*Hj)^Ns)Ij } + Ir)$$
$$Ir = (intensity of reflection map) + (ray trace)$$

\item[5]{这是一个漫反射照明模型,}
$$$$

\item[6]{这是一个漫反射照明模型,}
$$color=KaIa+ Kd { SUM j=1..ls, (N * Lj)Ij }$$

\item[7]{这是一个漫反射照明模型,}
$$color=KaIa+ Kd { SUM j=1..ls, (N * Lj)Ij }$$

\item[8]{这是一个漫反射照明模型,}
$$color=KaIa+ Kd { SUM j=1..ls, (N * Lj)Ij }$$

\item[9]{这是一个漫反射照明模型,}
$$color=KaIa+ Kd { SUM j=1..ls, (N * Lj)Ij }$$

\item[10]{ 这是一个漫反射照明模型,}
$$color=KaIa+ Kd { SUM j=1..ls, (N * Lj)Ij }$$
\end{itemize}


\item[(1)]{\bm{$Ns \hspace{10pt} 10.000000$}}\\
指定材质的反射指数,定义了反射高光度。 exponent是反射指数值,该值越高则高光越密集,一般取值范围在$[0,1000]$。 

\item[(2)]{\bm{$Ni \hspace{10pt} 1.500000$}} \\
指定材质表面的光密度(即折射值),取值范围为$[0.001,10]$。若取值为1.0则光在通过物体的时候不发生弯曲。玻璃的折射率为1.5。取值小于1.0的时候可能会产生奇怪的结果不推荐。

\item[(3)]{\bm{$d \hspace{10pt} 1.000000$}}\\
表示物体融入背景的数量,取值范围为$[0.0,1.0]$,取值为1.0表示完全不透明,取值为0.0时表示完全透明。

\end{itemize}


\subsection{kd-tree的介绍}


\section{步骤}
导入obj$\to$绘制物体$\to$判断纹理文件是否存在$\to$导入纹理$\to$贴纹理$\to$建立kd-tree
\end{document}